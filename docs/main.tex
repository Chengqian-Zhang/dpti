%\documentclass[aps,pre,preprint,a4paper,onecolumn]{revtex4}
\documentclass[aps,pre,a4paper,showkeys,fleqn]{revtex4}
% \documentclass[aps, pre, reprint,unsortedaddress,a4paper,onecolumn, showkeys]{revtex4}

\usepackage[]{amsmath}
\usepackage{amssymb}
\usepackage[caption = false]{subfig}
\usepackage[dvips]{graphicx}
\usepackage{color}
\usepackage{tabularx}
\usepackage{mathtools}
\usepackage{algpseudocode}
\usepackage{enumitem}
\usepackage{multirow}
\usepackage{epstopdf}  
\usepackage{hyperref}
\usepackage{footnote}
\usepackage[hang,flushmargin]{footmisc}
\usepackage{bm}
\usepackage{xr}

\newcommand*\chem[1]{\ensuremath{\mathrm{#1}}}

\newcommand{\recheck}[1]{{\color{red} #1}}
% \newcommand{\redc}[1]{{\color{red} #1}}
\newcommand{\redc}[1]{{\color{red} #1}}
\newcommand{\bluec}[1]{{\color{blue} #1}}
\newcommand{\greenc}[1]{{\color{green} #1}}

\begin{document}

\title{
Write down
}

\maketitle

\section{Ideal Einstein molecules}
Consider the lattice that is defined by the first atom in the system, saying the oxygen of the first water molecule.
The number of atoms is denoted by $N$, the number of oxygen atoms is denoted by $N_O$ and the number of hydrogen atoms is denoted by $N_H$.
The partition function is written as
\begin{align*}
  Q & = \frac{1}{h^{3N}}
      \frac{(N_O-1)!N_h!}{N_O!N_h!}
      \int e^{-\beta\sum_i\frac{p_i^2}{2m_i}} dp
      \int dr_0\int e^{-\beta\sum_i k(r_i - r_{i0})^2} dr \\
    &=
      \frac{1}{h^{3N}}\frac{1}{N_O}
      (2\pi m k_BT)^{\frac{3N}{2}}
      {V}
      \Big(\frac{\pi k_BT}{k}\Big)^{\frac{3N-3}2} \\
    &=
      \frac{V}{N_O}
      \Big( \frac{2\pi m k_BT}{h^2} \Big)^{\frac{3N}{2}}
      \Big( \frac{\pi k_BT}{k} \Big)^{\frac{3N-3}{2}}
\end{align*}
We let
\begin{align*}
  \Lambda_p = \Big(\frac{h^2}{2\pi m k_BT}\Big)^{\frac12}, \quad \Lambda_k = \Big(\frac{\pi k_BT}{k}\Big)^{\frac12}
\end{align*}
Then
\begin{align*}
  A
  &=
    -k_BT \ln Q \\
  &=
    -k_BT \ln \frac1{\rho_O} - k_BT \ln \Lambda_p^{-3N} - k_BT \ln \Lambda_s^{3N-3} \\
  &=
    k_BT \ln {\rho_O} + Nk_BT \ln \Lambda_p^3 - (N-1) k_BT \ln \Lambda_s^3
\end{align*}


\section{Ideal gas with springs}
Consider the partition function
\begin{align*}
  Q
  &=
    \frac{1}{h^{3N}}\frac{1}{N_O! N_H!}
    \frac{N_H!}{2^{N_O}}
    \int e^{-\beta\sum_i\frac{p_i^2}{2m_i}} dp
    \int e^{-\beta \sum_i [k(r^O_i-r^{H_1}_i)^2 + k(r^O_i-r^{H_1}_i)^2]} dr^{H_1}dr^{H_2} dr^O\\
  &=
    \frac{1}{h^{3N}}\frac{1}{N_O! 2^{N_O}}
    (2\pi m k_BT)^{\frac{3N}{2}}
    V^{N_O}
    \Big(\frac{\pi k_BT}{k}\Big)^{\frac{3N_H}2} \\
  &=
    \frac{1}{N_O! 2^{N_O}}
    \Lambda_p^{-3N}
    V^{N_O}
    \Lambda_k^{3N_H}
\end{align*}
Then
\begin{align*}
  A
  &=
    -k_BT\ln Q\\
  &=
    Nk_BT\ln \Lambda_p^3
    -N_Ok_BT\ln V
    -N_Hk_BT\ln\Lambda_k^3 
    + k_BT(N_O\ln N_O - N_O + \frac12\ln 2\pi N_O)
    + N_H k_BT\ln \sqrt2 \\
  &=
    Nk_BT\ln \Lambda_p^3
    + N_Ok_BT\ln\rho_O
    + N_Hk_BT\ln(\sqrt2/\Lambda_k^3)
    + k_BT( - N_O + \frac12\ln 2\pi N_O)
\end{align*}


\bibliography{ref}{}
\bibliographystyle{unsrt}
% \bibliographystyle{apsrev4-1} % Tell bibtex which bibliography style to use

\end{document}
