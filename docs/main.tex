%\documentclass[aps,pre,preprint,a4paper,onecolumn]{revtex4}
\documentclass[aps,pre,a4paper,showkeys,fleqn]{revtex4}
% \documentclass[aps, pre, reprint,unsortedaddress,a4paper,onecolumn, showkeys]{revtex4}

\usepackage[]{amsmath}
\usepackage{amssymb}
\usepackage[caption = false]{subfig}
\usepackage[dvips]{graphicx}
\usepackage{color}
\usepackage{tabularx}
\usepackage{mathtools}
\usepackage{algpseudocode}
\usepackage{enumitem}
\usepackage{multirow}
\usepackage{epstopdf}  
\usepackage{hyperref}
\usepackage{footnote}
\usepackage[hang,flushmargin]{footmisc}
\usepackage{bm}
\usepackage{xr}

\newcommand*\chem[1]{\ensuremath{\mathrm{#1}}}

\newcommand{\recheck}[1]{{\color{red} #1}}
% \newcommand{\redc}[1]{{\color{red} #1}}
\newcommand{\redc}[1]{{\color{red} #1}}
\newcommand{\bluec}[1]{{\color{blue} #1}}
\newcommand{\greenc}[1]{{\color{green} #1}}

\newcommand{\com}[0]{{\textrm{CM}}}

\begin{document}

\title{
Write down
}

\maketitle

\section{Ideal Einstein crystal with COM constrained}

The partition function for the free Einstein crystal is
\begin{align*}
  Q & =
      \frac{1}{h^{3N}}
      \frac{1}{N!} N!
      \int e^{-\beta\sum_i\frac{p_i^2}{2m_i}} dp
      \int e^{-\beta\sum_i \frac 12k(r_i - r_{i0})^2} dr \\
    &= 
      P \int e^{-\beta\sum_i \frac 12k(r_i - r_{i0})^2} dr \\
    &=
      P \Big( \frac{2\pi k_BT}{k} \Big)^{\frac{3N}2}
\end{align*}
where $r_{i0}$ is the equilibrium position of atom $i$, and $P$ is the momentum part of the partition function.
We let $\epsilon = \frac1N \sum_i r_i - r_{i0}$,
\begin{align*}
  Q & =
      P \int e^{-\beta\sum_i \frac 12k(r_i - r_{i0})^2} dr \\
    & =
      P \int e^{-\beta\sum_i \frac 12k(r_i - r_{i0})^2} \delta\big( \epsilon - \frac1N \sum_i (r_i - r_{i0}) \big) dr d\epsilon 
\end{align*}
Now consider $\hat r_i = r_i - \epsilon$, we have
\begin{align}\label{eqn:1}
  Q & =
      P \int e^{-\beta\sum_i \frac 12k(\hat r_i - r_{i0} + \epsilon)^2} \delta\big( \frac1N \sum_i (\hat r_i - r_{i0}) \big) d\hat r d\epsilon 
\end{align}
We have the observation that $ \frac 1N \sum (\hat r_i  - r_{i0}) = \frac 1N \sum ( r_i  - r_{i0}) - \epsilon = 0$, then
\begin{align*}
  \sum_i (\hat r_i - r_{i0} + \epsilon)^2
  &=
    \sum_i (\hat r_i - r_{i0})^2 + 2\epsilon\sum_i (\hat r_i - r_{i0}) + N\epsilon^2 \\
  &=
    \sum_i (\hat r_i - r_{i0})^2 + N\epsilon^2 
\end{align*}
Inserting this relation to \eqref{eqn:1}, we have
\begin{align}\nonumber
  Q
  & =
    P \int e^{-\beta\sum_i \frac 12k(\hat r_i - r_{i0})^2} e^{-\beta \frac12 kN \epsilon^2} \delta\big( \frac1N \sum_i (\hat r_i - r_{i0}) \big) d\hat r d\epsilon \\
  &=
    P \Big( \frac{2\pi k_BT}k  \Big)^{\frac32} N^{-\frac32}
    \int e^{-\beta\sum_i \frac 12k(\hat r_i - r_{i0})^2} \delta\big( \frac1N \sum_i (\hat r_i - r_{i0}) \big) d\hat r
\end{align}
% and consider the transformation $\hat r_i = r_i - \epsilon$.
Now consider the constrain of the COM, we have the modified partition function
\begin{align*}
  Q_\com
  &= 
    {P_\com}
    \int e^{-\beta\sum_i \frac 12k(r_i - r_{i0})^2} \delta\big(\frac1N\sum_i (r_i -  r_{i0})\big) dr 
\end{align*}
where $P_\com$ is the integration of the COM-constrained momentum part.
% We let $\epsilon = \frac1N \sum_i r_i - r_{i0}$, and consider the transformation $\hat r_i = r_i - \epsilon$.
% It is easily shown that $\sum_i \hat r_i - r_{i0} = 0$.
% We have, 
% \begin{align*}
%   Q_\com
%   &= 
%     {P_\com}
%     \int e^{-\beta\sum_i \frac 12k(r_i - r_{i0})^2}
%     \delta\big(\frac1N\sum_i r_i - \frac1N\sum_i r_{i0}\big)
%     dr \\
%   &= 
%     {P_\com}
%     \int e^{-\beta\sum_i \frac 12k(\hat r_i - r_{i0} + \epsilon)^2}
%     \delta\big(\frac1N\sum_i \hat r_i - \frac1N\sum_i r_{i0} + \epsilon\big)
%     J \: d\hat r 
% \end{align*}
% where $J$ is the Jacobian of the transformation.
% We notice that
% \begin{align*}
%   \sum_i (\hat r_i - r_{i0} - \epsilon)^2
%   &=
%   \sum_i (\hat r_i - r_{i0})^2 + 2\epsilon \sum_i (\hat r_i - r_{i0}) + N \epsilon^2 \\
%   &=
% \end{align*}
Then we have
\begin{align}
  \frac{Q}{Q_\com} = \frac{P}{P_\com}\Big( \frac{2\pi k_BT}k  \Big)^{\frac32} N^{-\frac32}
\end{align}
so
\begin{align}
  A - A_\com = -k_BT \ln(P/P_\com) - k_BT \ln \Lambda_k^3 + k_BT \ln N^{3/2}
\end{align}

\section{Translational invariant system with CM constrained}
The partition function for translational invariant system
\begin{align*}
  Q = P \int e^{-\beta U(\bm r)}d \bm r
\end{align*}
where $U(\bm r) = U(\bm r - \bm r_0)$ for any $\bm r_0$.
\begin{align*}
  Q
  &=
    P \int e^{-\beta U(\bm r)} \delta(\sum_i c_i (\bm r_i - \bm r_0)) d\bm r d\bm r_0 \\
  &=
    P\int d\bm r_0 \int e^{-\beta U(\bm r)} \delta(\sum_i c_i \bm r_i) d\bm r\\
  &=
    P\frac{V}{N} \int e^{-\beta U(\bm r)} \delta(\sum_i c_i \bm r_i) d\bm r
\end{align*}
The last equation holds because the volume ``accessible to the CM'' is $V/N$, see [Vega, et.al.~JPCM].
\begin{align*}
  Q_\com
  &=
    P_\com\int e^{-\beta U(\bm r)} \delta(\sum_i c_i \bm r_i) d\bm r
\end{align*}
Therefore
\begin{align*}
  \frac{Q}{Q_\com} = \frac{P}{P_\com}\frac{V}{N}
\end{align*}
and
\begin{align}
  A - A_\com = -k_BT\ln(P / P_\com) + k_BT\ln \rho
\end{align}

\section{Ideal Einstein molecules}
Consider the lattice that is defined by the first atom in the system, saying the oxygen of the first water molecule.
The number of atoms is denoted by $N$, the number of oxygen atoms is denoted by $N_O$ and the number of hydrogen atoms is denoted by $N_H$.
The partition function is written as
\begin{align*}
  Q & = \frac{1}{h^{3N}}
      \frac{(N_O-1)!N_h!}{N_O!N_h!}
      \int e^{-\beta\sum_i\frac{p_i^2}{2m_i}} dp
      \int dr_0\int e^{-\beta\sum_i \frac 12k(r_i - r_{i0})^2} dr \\
    &=
      \frac{1}{h^{3N}}\frac{1}{N_O}
      (2\pi m k_BT)^{\frac{3N}{2}}
      {V}
      \Big(\frac{2\pi k_BT}{k}\Big)^{\frac{3N-3}2} \\
    &=
      \frac{V}{N_O}
      \Big( \frac{2\pi m k_BT}{h^2} \Big)^{\frac{3N}{2}}
      \Big( \frac{2\pi k_BT}{k} \Big)^{\frac{3N-3}{2}}
\end{align*}
We let
\begin{align*}
  \Lambda_p = \Big(\frac{h^2}{2\pi m k_BT}\Big)^{\frac12}, \quad \Lambda_k = \Big(\frac{2\pi k_BT}{k}\Big)^{\frac12}
\end{align*}
Then
\begin{align*}
  A
  &=
    -k_BT \ln Q \\
  &=
    -k_BT \ln \frac1{\rho_O} - k_BT \ln \Lambda_p^{-3N} - k_BT \ln \Lambda_s^{3N-3} \\
  &=
    k_BT \ln {\rho_O} + Nk_BT \ln \Lambda_p^3 - (N-1) k_BT \ln \Lambda_s^3
\end{align*}


\section{Ideal gas with springs}
Consider the partition function
\begin{align*}
  Q
  &=
    \frac{1}{h^{3N}}\frac{1}{N_O! N_H!}
    \frac{N_H!}{2^{N_O}}
    \int e^{-\beta\sum_i\frac{p_i^2}{2m_i}} dp
    \int e^{-\beta \sum_i [\,k(\vert r^O_i-r^{H_1}_i\vert - r_0)^2 + k(\vert r^O_i-r^{H_1}_i\vert - r_0)^2]} dr^{H_1}dr^{H_2} dr^O\\
  &=
    \frac{1}{h^{3N}}\frac{1}{N_O! 2^{N_O}}
    (2\pi m k_BT)^{\frac{3N}{2}}
    V^{N_O}
    % \Big(\frac{\pi k_BT}{k}\Big)^{\frac{3N_H}2} 
    \Big(\int e^{-\beta k(r - r_0)^2 } d\bm r\Big)^{N_H} \\
  &=
    \frac{1}{N_O! 2^{N_O}}
    \Lambda_p^{-3N}
    V^{N_O}
    % \Lambda_k^{3N_H}
    (V_H)^{N_H}
\end{align*}
To compute
\begin{align*}
  V_H = \int e^{-\beta k(r - r_0)^2 } d\bm r
  &= 
    \int_0^\pi\int_0^{2\pi}\int_0^\infty e^{-\beta k(r - r_0)^2 } r^2\sin\phi dr d\theta d\phi\\
  &=
    4\pi\int e^{-\beta k(r - r_0)^2 } r^2 dr\\
  & =
    4\pi\int r^2 e^{-\frac{(r - r_0)^2}{2 (\sqrt{k_BT/2k})^2}}dr\\
  & =
    4\pi \Big(\frac{\pi k_B T}{k}\Big)^\frac12
    \int r^2
    \Big(\frac{k}{\pi k_B T}\Big)^\frac12
    e^{-\frac{(r - r_0)^2}{2 (\sqrt{k_BT/2k})^2}}dr\\
  &=
    4\pi \Big(\frac{\pi k_B T}{k}\Big)^\frac12 (r_0^2 + \frac{k_BT}{2k})
\end{align*}
Then
\begin{align*}
  A
  &=
    -k_BT\ln Q\\
  &=
    Nk_BT\ln \Lambda_p^3
    -N_Ok_BT\ln V
    -N_Hk_BT\ln V_H
    + k_BT(N_O\ln N_O - N_O + \frac12\ln 2\pi N_O)
    + N_H k_BT\ln \sqrt2 \\
  &=
    Nk_BT\ln \Lambda_p^3
    + N_Ok_BT\ln\rho_O
    + N_Hk_BT\ln(\sqrt2/V_H)
    + k_BT( - N_O + \frac12\ln 2\pi N_O)
\end{align*}

The pressure of the system is computed by (Tuckerman 4.6.55 and 4.6.63)
\begin{align*}
  P
  = \frac{Nk_BT}{V} +
  \frac{1}{3V}
  \Big\langle
  \sum_i \bm r_i\cdot\bm F_i  
  \Big\rangle
  = \frac{Nk_BT}{V} +
  \frac{1}{6V}
  \Big\langle
  \sum_{i\neq j} \bm r_{ij}\cdot\bm F_{ij}
  \Big\rangle
\end{align*}
The spring force is given by
\begin{align*}
  \bm F_{ij}
  &=
    -\nabla_{\bm r_{ij}} [k (r_{ij} - r_0)^2] \\
  &=
    -2k(r_{ij} - r_0) \frac{\bm r_{ij}}{r_{ij}}
\end{align*}
thus
\begin{align*}
  \langle \bm r_{ij}\cdot\bm F_{ij} \rangle
  &=
    \langle -2k(r_{ij} - r_0)r_{ij} \rangle \\
  &=
    -2k \int r(r - r_0) \frac{1}{V_H} e^{-\beta k(r - r_0)^2} d\bm r \\
  &=
    -2k \frac{4\pi}{V_H} \sqrt{2\pi\sigma^2} \int r^3(r - r_0)\frac{1}{\sqrt{2\pi\sigma^2}} e^{-\beta k(r - r_0)^2} dr\\
  &=
    -2k \frac{4\pi}{V_H} \sqrt{2\pi\sigma^2} (r_0^4 + 6 r_0^2\sigma^2 + 3 \sigma^4 - r_0^4 - 3r_0^2\sigma^2 )\\
  &=
    -2k \frac{12\pi}{V_H} \sqrt{2\pi\sigma^2} \, ( r_0^2 + \sigma^2)\sigma^2
  % -2k\sigma^2
  % =
  % -2k \times \frac{k_BT}{2k}
  % = -k_BT
\end{align*}
Notice that $V_H = 4\pi\sqrt{2\pi\sigma^2} (r_0^2 + \sigma^2)$
\begin{align*}
  \langle \bm r_{ij}\cdot\bm F_{ij} \rangle
  &= -6k \sigma^2 = -3 k_BT
\end{align*}
so
\begin{align*}
  P = \frac{N_O k_BT}{V} + \frac{N_H k_BT}{2V}
\end{align*}

\bibliography{ref}{}
\bibliographystyle{unsrt}
% \bibliographystyle{apsrev4-1} % Tell bibtex which bibliography style to use

\end{document}
